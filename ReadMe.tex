\documentclass[a4paper,10pt]{article}
\begin{document}
\begin{center}
\textbf{Read me}
\end{center}
\begin{itemize}
\item[1]Once we get the Alloy up and running we open a model Alloy1.als.
\item[2]Select a command to execute by choosing one from the `Execute' menu the menu bar.
\item[3]Execute the selected command with the 'Execute' icon or by choosing Execute from the Tools menu. You will be told ``counterexample found. Assertion is invalid". Click on it, and a new window will open up visualizing the counterexample.
\item[4]You can also browse a tree structure of the model by selecting the ``tree" button from the toolbar. The tree view allows you to see how the signatures and fields in your model evaluated in this particular solution. You can return to the diagram view by choosing the ``viz" button from the toolbar.
\item[5]You can make the visualization more readable by customizing the layout. You do this by clicking the ``Theme" button from the toolbar. You can adjust what relations and sets are shown, and the style of their presentation. Here is a cleaner customization by clicking on the ``contents" relationship and setting its ``Show as Arc" checkbox to false.
\item[6]Similarly for the models Alloy2.als, Alloy3.als, Alloy4.als, Alloy5.als which on executing shows ``Instance found. Predicate is consistence". Click on it and a new window will open up visualizing the Instance.
\end{itemize}
\end{document}